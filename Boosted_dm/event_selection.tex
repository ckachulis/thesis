\section{Event Selection}
\label{sec:event_selection}
A cut based selection is used to select electron elastic-scatter-like events.  The selection begins with the FCFV sample, which consists of all events which pass the Fully-Contained event reduction, plus four additional cuts: wall$>$200 cm, which defines the fiducial volume, nhitac$<$16, evis$>$30 MeV, and if the event is 1-ring elike, the final cut amome$>$100 MeV.  From this FCFV sample, 4 analysis cuts are applied: \\
\\
\indent 
1. 1-ring\\
\indent
2. e-like\\
\indent
3. 0 decay electrons\\
\indent
4. 0 tagged neutrons\\
\\
 
The first two cuts search for the single relativistic scatter from an electron elastic scatter, while the final two cuts remove events with any indication of a nuclear interaction.  The number of events found in all SK-IV after each cut is shown for three different energy ranges in \ref{tab:cut_counts}   The neutron tagging ut plays an important role in background reduction, particularly in the high energy sample, as can be seen in \cref{tab:cut_counts}.  It's inclusion limits the current analysis to SK-IV only, since neutron tagging is not possible on SK-I-III data.
\begin{table*}
\begin{tabular}{lccc}
\hline \hline
& Evis$<$1.33 GeV & 1.33 GeV$<$Evis$<$20 GeV & 20 GeV$<$ Evis \\
\hline
FCFV & 15206 & 4908 & 97 \\
and single ring & 11367 & 2868 & 53 \\
and e-like & 5655 & 1514 & 53 \\
and 0 decay-e & 5176 & 1134 & 17 \\
and 0 tagged neutrons & 4132 & 683 & 4 \\
\hline 
\end{tabular}
\caption{Number of events in SK4 data passing each cut for 3 energy ranges.}
\label{tab:cut_counts}
\end{table*}
