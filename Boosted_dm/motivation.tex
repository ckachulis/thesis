\section{Introduction}
\label{sec:motivation}
There is ample evidence for the existence of dark matter \cite{Zwicky:1933gu,Rubin:1980dm,Begeman:1991iy,Blumenthal:1984eu}, and it's abundance has been well measured to account for about 25\% of the energy density of the universe \cite{PlanckCollaboration:2016cf,Hinshaw:2013dd}.  However, beyond it's existence and abundance, little else is known about the properties of dark matter.  A promising possible dark matter particle is the Weakly Interacting Dark Matter Particle (WIMP), but thus far attempts at it's direct detection have left such a particle undiscovered \cite{Collaboration:2013vx,TheCDMSIICollaboration:2010dr,Aprile:2012kx}.  Indirect detection of dark matter through the detection of it's standard model annihilation products and detection of dark matter produced at particle accelerators have thus far similarly produced null results \cite{Desai:2004pq,Rameez:2015nvz,Adrian-Martinez:2013ayv,Khachatryan:2014rra,Aad:2015zva,Goodman:2010ku}.  
Boosted dark matter has been recently proposed in \cite{Agashe:2014yua} and discussed as an alternative to the standard WIMP paradigm \cite{Cherry:2015gw,Bhattacharya:2014yha,Kopp:2015gp,Kong:2015jb,Berger:2014hq}.   As opposed to WIMP models where the dominant dark matter particle couples directly to standard matter, in Boosted dark matter models the dominant dark matter particle instead couples to a lighter dark matter particle which in turn couples to standard matter.  The lighter dark matter particle could be produced through annihilation, semi-annihilation, or decay of the heavier dark matter particle, and if the mass difference between the dominant heavy dark matter and lighter sub-dominant dark matter is large enough these light dark matter particles will be relativistic.  Reference \cite{Agashe:2014yua} introduces an example model where the heavy dark matter particle annihilates into the lighter dark matter particle, which couples to standard model particles through the exchange of a dark photon.
Such dark matter scenarios can be searched for by looking for the interaction of the boosted dark matter particle with standard matter.  These interactions could be either with electrons, or nucleons.  This technote reports a search for interactions of boosted dark matter particles with electrons in Super Kamiokande (SK), with the boosted dark matter originating either from the Galactic Center or the sun.  In SK these interactions appear as elastically scattered electrons pointing back to either the Galactic Center or the sun.  This search is performed on 2,628.1 days of SK-IV data, which corresponds to 161.9 kT-years. 




