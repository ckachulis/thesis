\chapter{Data Processing}
\label{ch:data_processing}
\graphicspath{{data_processing/}}
Super-Kamiokande records around $10^6$ events per day.  However, the vast majority of these events are either low energy radioactive backgrounds (~11 Hz) or cosmic ray muons (~3 Hz).  For comparison, the rate of atmospheric neutrino interactions in SK is about 10 per day.  Data reduction processes are used to select only the reatively small number of interesting physics events.  In the high energy range ($>100 MeV$), three samples are used, each selected by a separate data reduction process:
\begin{itemize}
\item Fully Contained (FC):  These are events with activity in the ID and no activity in the OD.  These events are the best reconstructed, since all the energy of the event is contained within the ID.
\item Partically Contained (PC):  These are events with activity in both the ID and OD, but where it has been determined that this is due to a particle starting in the ID and exiting into the OD.  These events are almost all muons, since most electromagnetic showers will not have the energy to puncture from the FV into the OD.  These events are more poorly reconstructed than FC events, since much of their energy can be deposited outside the ID, and even outside the OD.
\item Upward Going Muons (UPMU):  These are events with activity in both the ID and the OD, where the event starts in the OD, but where the event is coming from below the detector.  Since these events come from below, the Earth itself is acting as a shield for the detector.  The events are thus due to neutrinos interacting in the rock below the detector.
\end{itemize}

In this chapter, I will discuss all three of these reduction samples.
\section{Fully Contained Reduction}
\label{sec:FC}
The FC data reduction consists of five steps, labeled FC1-FC5.  Combined, they select around 10 events per day from the $10^6$ events recorded by SK, with an efficiency for selecting events which originate in the FV of about 98\%.  In the following descriptions, events which pass the cuts listed are the events which make up the FC sample.
\subsection{FC1}
FC1 consists of a two simple cuts:
\begin{itemize}
\item The number of pe in a 300 ns sliding time window in the ID is greater than 200 (100 for SK-II)
\item There are fewer than 50 (55 for SK-IV) OD tubes hit.
\end{itemize}
The first cut removes low energy radioactive background events and solar neutrinos, while the second cut removes obvious cosmic ray muon and PC or UPMU events.  Of the $10^6$ events recorded by SK each day, about 2,500 pass FC1.

\subsection{FC2}
FC2 also consists of two simple cuts:
\begin{itemize}
\item No ID PMT can be responsible for greater than 50\% of the pe observed in the ID
\item If the number of pe in the ID is less than 100,000 (50,000 for SK-II), there must be fewer than 25 hits in the OD (30 for SK-IV)
\end{itemize}
This first cut removes electrical noise events where a single large pulse on one PMT accounts for most of the measured ID charge.  The second cut is a stricter version of the second FC1 cut, but allows for situations where a very high energy event which is contained to the ID may have more hits than lower energy events in the ID, due to electrical cross-talk between channels.  Of the about 2,500 events which pass FC1 each day, about 500 pass FC2.

\subsection{FC3}
\section{Partially Contained Reduction}
The partially contained reduction searches for muons that originated in the ID but were able to penetrate into the OD.  Similarly to the FC reduction, this is done in a sequence of five steps, labeled PC1-PC5.  The PC reduction selects about two events every three days from the $10^6$ events recorded by SK.  Between SK-II and SK-III, the top, bottom, and barrel of the OD were optically separated from one another.  This segmentation allowed for a more efficient PC selection from SK-III onward.  I will describe here the PC selection in SK-III and SK-IV, description of the selection for early periods can be in \cite{Dufour:2009ljt}. 
\section{Upward Going Muon Reduction}
\label{sec:upmu}