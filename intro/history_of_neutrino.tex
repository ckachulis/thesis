\section{A Brief History of the Neutrino}
While neutrinos have been around for about 13.8 billion years \cite{PlanckCollaboration:2016cf,Hinshaw:2013dd}, and humans for a slightly shorter but still respectable about 200,000 years, humans have only been aware of the neutrino for the relatively short period of about 90 years.  The existence  neutrino was first proposed by Wolfgang Pauli in a letter to a meeting in Tubingen, Germany, in 1930, as a possible explanation for the observed continuous beta decay spectrum \cite{Brown:1978pb}.  In this letter Pauli wrote
\begin{quote}
...there could exist in the nuclei electrically neutral particles that I wish to call neutrons, which have spin 1/2 and obey the exclusion principle, and additionally differ from light quanta in that they do not travel with the velocity of light.  The mass of the neutron must be of the same order of magnitude as the electron mass and, in any case, not larger than 0.01 proton mass.--The continuous $\beta$-spectrum would then become understandable by the assumption that in $\beta$ decay a neutron is emitted together with the electron, in such a way that the sum of the energies of neutron and electron is constant. \cite{Brown:1978pb}
\end{quote} 
This first proposal was tentative, and it is interesting to note that Pauli first used the name "neutron," the particle we today know as the neutron having not yet been discovered.  In 1932, the heavy neutron we know today was discovered by James Chadwick, and so the light neutral particle first proposed by Pauli was renamed the "neutrino" by Enrico Fermi.  In 1934, Fermi published a quantitative theory of $\beta$-decay which included neutrinos as an assumed component \cite{Fermi:1934hr}, and predicted the continuous beta-decay spectrum.  \par
Continued measurements of the beta-decay spectrum and observations of new meson decays using nuclear track emulsion strengthened the evidence in the existence of the neutrino.  However, it was not until the early 1950's that the neutrino was explicitly observed.  Fred Reines and Clyde Cowan placed a large liquid scintillator detector near a nuclear reactor in Hanford, Washington.  They hoped to observe neutrinos produced by the reactor interacting with the protons of the liquid scintillator through inverse beta decay, $p+\bar{\nu}\rightarrow n+e^+$.  They searched for events with a delayed coincidence between the scintillation light due to the positron and the $\gamma$ produced by the capture of the neutron on Cadmium with which the liquid scintillator had been loaded.  They noted a change in the measured event rate when the reactor was on versus when it was off, which agreed well with the predicted neutrino event rate \cite{Reines:1953pu}.  They soon confirmed the result with an upgraded experiment as Savanah River \cite{Cowan:1992xc,Reines:1956rs}.    