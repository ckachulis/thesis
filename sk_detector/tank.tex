\section{The Tank}
\label{sec:tank}
The main component of the SK detector is a cylindrical stainless steel tank, with a diameter of 39 m and a height of 42 m, which is filled with about 50 ktons of water.  The tank is segmented into an inner detector (ID), with a diameter 33.8 m and a height of 36.2 m, which hold 32 ktons of water, and an outer detector (OD) which is the region of the tank outside the ID.  The ID is the primary detector used for most physics analyses, while the OD is primarily used as an active cosmic ray veto.  The ID and OD are separated from one another by a cylindrical PMT support structure.  On the inner surface of the support structure, 11,146 inward-facing 20 inch PMTs, giving a coverage of about 40\% are mounted to observe activity in the ID (for SK-II half as many PMTs were used in the ID).  The outer surface of the support structure is instrumented with 1885 outward-facing 8 inch PMTs to observe the OD.  Lightproof Tyvek sheeting on both surfaces of the PMT structure optically separates the ID from the OD.  It also results in a 55 cm this dead space between the ID and the OD, from which light cannot escape.  The Tyvek sheeting is black on the side facing into the ID, in order to reduce reflections which would diminish reconstruction accuracy.  On the side facing the OD, conversely, the Tyvek sheeting is white, in order to increase reflections.  This is done to improve light collection efficiency in the OD, the compensate for its lower PMT coverage.  Scattered light in the OD is also much less problematic for physics goals compared to scattered light in the ID.  The structure of the detector is shown in \cref{fig:sk_detector_diagrams}

\begin{figure*}
\centering
\includegraphics[width=0.8\textwidth]{figures/sk_detector_under_mountain.jpg}
\caption{Structure of the SK detector. \cite{Fukuda:2002uc}}
\label{fig:sk_detector_diagrams} 
\end{figure*}

\begin{figure*}
\centering
\includegraphics[width=0.8\textwidth]{figures/pmt_support_structure.jpg}
\caption{PMT support structure. \cite{Fukuda:2002uc}}
\label{fig:pmt_support_structure} 
\end{figure*}